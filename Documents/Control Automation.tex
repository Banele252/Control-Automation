\documentclass{article}
\usepackage{graphicx} % Required for inserting images

\title{Control automation}
\author{banele mdluli}
\date{February 2026}

\begin{document}

\maketitle
\newpage
\section{Summary}
Control automation is a project to automate control performance. The automation will cover all processes associated with control performance. This includes frequent data extraction, logic implementation, and exception recording. Beyond control performance, an automation management console will be created to interact with various automation parts.

\section{Introduction}
Control performance is a crucial BAU (Business As Usual) activity aimed at mitigating risks associated with business processes that could result in financial loss. A control is a mechanism used to mitigate financial risk in a business through manual, semi-automated, or automated procedures. These procedures are performed using various tools tailored for risk mitigation. Control performance is the process of observing, recording and analysing control outcomes. Governance provides guidance on how control performance is done based on audit requirements and senior management expectations.

\section{Problem statement}
The current control performance method requires human resources to perform. In most cases, an analyst is required to obeserve, record and analyse the control output. Additionally, some control outcomes can require secondary analysis, whereby further analysis is done. There are three phases that make-up control performance: obeserve, record and analyse control outcome. The observe and record phase are repetitive, manually intensive and rarely change. While analysis will differ depending on the recorded outcome.\\
\\
Control performance dependency on human resources results in a cost in operational expenditure. The OpEx can be reduced using automation for observing and recording control outcomes and using an A.I agent to analyse an outcome.

\section{Objectives}
The project objectives are targets that must be reached for the project to be considered a success. These objectives are set to reduce the cost associated with control performance and enabling better re-allocation of resources for more critical activities. The objectives are as follows:
\begin{itemize}
    \item To create a platform that extracts data from a database using a predefined logic (control logic)
    \item To create a platform that records the information in a control outcome database.
    \item To create a platform that analysis the data that was recorded in a control outcome database saves it in the control outcome database.
\end{itemize}
\section{Requirements}
\subsection{Functional requirements}
\subsection{Non functional requirements}
\subsection{Technical requirements}
\section{Design}
\section{Methodology}
\section{Discussion}
\section{Conclusion}
\section{Appendix}
\subsection{Code}

\end{document}
